\documentclass[11pt]{article}

%\usepackage{graphics}
\usepackage{graphicx,amsmath,amssymb}
%\usepackage[spanish]{babel} 
\usepackage[utf8]{inputenc}
\textwidth 16.5cm
\textheight 25.0cm
\voffset -2.9cm
\hoffset -2.0cm

\begin{document}
\pagestyle{empty}

\begin{center}

{\Large {\bf Report of ECI 2019 Course: `` Introduction to Steganography and Watermarking ``}} \\

\bigskip
{\large \bf Assignment E.316-N}
\end{center}

\begin{center}
Acha Francisco, Caldo Juan Pablo, Cardenas Rodrigo, Castagna Franco and Remedi Elias.
\end{center}

%% ABSTRACT!! (At the end)

\section{Introduction}

%Define steganography
Steganography is the procedure of insert information inside a data source without changing its perceptual quality.
Digital steganography uses digital data sources as a cover for hidden information. Examples of digital covers
are digital text files, image files and sound files among others.  

%Introduce context for steganographic methods on images
In particular for digital image based steganography the pixel intensity is usually used for encoding information [REF]
but other approaches are also widely used such as embedding information in the frequency domain.

%Review steganography over images (main features)
%Tell about software available for digital image-steganography
There are available many software tools 

%Describe goals of current assignment
In this report, five steganographic tools that hides text into a digital image were chosen to perform an assessment 
in terms of imperceptibility of the stego-image, capacity and robustness.

\section{Materials and methods}

\subsection{Dataset}

%Dataset characterization
Since we want to evaluate performance of steganographic tools that hide text into an image, a image dataset is needed.
We built a dataset containg images 20 of four types: N-type (landscapes and open nature), S-type (still life), P-type (portraits) and
T-type (text). The complete dataset is then 80 images in total. N, S, P-type images were obtained and selected from Google images search engine queries.
Namely, keywords for queries were \textit{landscapes}, \textit{still life} and \textit{portrait} respectively.
Right usage for the images was selected such that results were labeled for noncommercial reuse, and size of the images was set in 
medium [NOTA AL PIE DE LA FECHA]. Text images were collected from research papers by exporting pages as jpeg images.
Table [REF] summarizes some basic features of the dataset used such as mean image size and mean file size. All the
images in the dataset were stored as jpeg format. DECIR AHORA EL TAMAÑO MEDIO, Y LA MEMORIA DE CADA IMAGEN
MOSTRAR UN EJEMPLO DE CADA TIPO



%Briefing of features for selected softwares
\subsection{Description of selected software}

\subsubsection{\textit{OutGuess (v. 0.2)}}
\textit{OutGuess} is  a  universal  steganographic  tool  that  allows  the  insertion  of  hidden information  into  the  
redundant  bits of data sources.  The nature of the data source is irrelevant to the core of outguess.  The program relies on  
data  specific  handlers  that will  extract  redundant  bits  and write them back after modification. Currently only the PPM, 
PNM, and JPEG image formats are supported, although outguess could use  any  kind  of data, as long as a handler were provided.
\textit{OutGuess} uses  a  generic  iterator  object  to  select  which bits in the data should be modified.  A seed can be used 
to modify the behavior of the iterator. It  is  embedded  in the data along with the rest of the message.  
By altering the seed, outguess tries to find a sequence of bits that minimizes the number of changes in the data that have to be 
made.

\textitA bias is introduced that favors the modification of bits that were extracted from a  high value, and tries to avoid the 
modification of bits that were extracted from a low value.

\textit Additionally,  {OutGuess}  allows  for the hiding of two distinct messages in the data, thus providing plausible 
deniablity.  It keeps track  of  the  bits  that  have  been  modified previously  and  locks  them.   A  (23,12,7)  Golay  code  
is used for error correction to tolerate collisions on locked bits.  Artifical errors are introduced  to  avoid  modifying bits 
that have a high bias.

\subsubsection{\textit{}}

\subsubsection{\textit{StegHide (v. 0.5.1)}}
\textit{StegHide} is an open source steganographic software that allows hide text using image or sound files as covers. (REF A LA PAGINA)
It supports JPEG, BMP, WAV and AU file formats as cover files. 
\textit{StegHide} performs steganography by means of a graph-theoretic approach. Data to be embedded is compressed and encrypted,
Then a pseudo-random sequence of postions of pixels is created. On this positions secret data will be embedded. Then a
graph-theoretic  matching  algorithm finds pairs of positions on the remaining pixels such that exchanging their values has
the effect of embedding the corresponding  part of the secret data. If there are not enough pixels with values that can be used
to embed the data by exchanging, values are overwroute. This way, most of the embedding  is  done  by  exchanging  pixel  values
and then the first-order statistics is marginally changed. A passphrase must be provided by the user for encryption and
pseudo-random generator initialization. The same passphrase must be provided for data extraction from stego-file. The default
encryption algorithm is Rijndael with a key of 128 bits altough others are available as well.



\subsubsection{\textit{}}

\subsubsection{\textit{SteganPEG}}

%Imperceptibility, capacity and robustness assessment
\subsection{Benchmarking}

Some criteria and metrics needs to be established in order to to benchmark the selected software.
In this section metrics for imperceptibility assessment are presented as well as criteria regarding capacity of storage for
hidden data and tests for robustness evaluation.

\subsubsection{Imperceptibility}

\subsubsection{Capacity}

\subsubsection{Robustness}


\section{Results}


%Show imperceptibility, capacity and robustness results
\subsection{Imperceptibility}

\subsection{Capacity}

\subsection{Robustness}


\section{Discussion and Conclusion}


\section{References}



\end{document}
